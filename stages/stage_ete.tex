\documentclass[12pt,a4paper]{article}

\usepackage{epsfig,amsmath,amssymb,amsbsy}
\usepackage[latin1]{inputenc}
\usepackage[cyr]{aeguill}
\usepackage[francais]{babel}
\setlength{\topmargin}{-0.2in}

\setlength{\parindent}{0.5cm}

\setlength{\parskip}{0.2cm plus 0.5ex minus 0.2ex}

\setlength{\partopsep}{0.0truept plus 0.0pt minus 0.0pt}

\setlength{\parsep}{0.0truept plus 0.0pt minus 0.0pt}

\setlength{\topsep}{0.0cm plus 0.0pt minus 0.0pt}

\addtolength{\hoffset}{-1.5truecm}

\addtolength{\textwidth}{2.8truecm}

\addtolength{\textheight}{3.9truecm}

\addtolength{\headheight}{-13truept}

\addtolength{\headsep}{-19truept}

\addtolength{\footskip}{20truept}

\date{}



\begin{document}

\centerline{\large STAGE R�MUN�R�}

\centerline{ \large MOD�LISATION MATH�MATIQUE EN BIOLOGIE MOL�CULAIRE}



P�riode : 5 juillet / 5 ao�t 2004

Lieu : Institut de recherches en math�matiques de Rennes (Universit� de Rennes 1, campus de Beaulieu)

Niveau : ma�trise de math�matiques, physique ou informatique. 

Connaissances requises : programmation MATLAB (ou SCILAB), notions d'�quations diff�rentielles ordinaires 

R�mun�ration : 530 euros/mois.   

Sujet du stage : Dynamique de r{\'e}seaux de g{\`e}nes : module de signalisation de NFkB

NF-kB est un facteur de transcription qui r�gule l'activit� d'un nombre important
de g�nes impliqu�s dans les processus de signalisation inter-cellulaires, de
croissance, de survie et de mort des cellules. Il fait partie d'un module
complexe de signalisation qui permet le changement de son niveau comme r�ponse � des
signaux ext�rieurs. On poss�de un mod�le dynamique de ce module qui consiste dans
un syst�me d'une centaine d'�quations diff�rentielles. Un solveur MATLAB est utilis�
pour trouver la solution du probl�me de conditions initiales pour ce syst�me. 
Le but de ce stage est dans un premier temps de v�rifier les informations utilis�es par le 
programme (r�actions bio-chimiques et valeurs de param�tres) et la  solution obtenue.
Dans un deuxi�me temps, on se propose d'identifier les variables lentes du syst�me
et de r�duire la complexit� de sa dynamique par des techniques de perturbations
singuli�res. Finalement, on se propose d'identifier des "boites noires" et d'exp�rimenter
une approche modulaire de la dynamique.   

Contact : Ovidiu Radulescu, IRMAR, Universit� de Rennes 1, Campus de Beaulieu, 
 
Bat. 22, Bureau 320, tel: 02 23 23 60 23

e-mail ovidiu.radulescu@univ-rennes1.fr



\end{document}
