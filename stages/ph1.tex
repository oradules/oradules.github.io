\documentclass[12pt,a4paper]{article}

\usepackage{epsfig,amsmath,amssymb,amsbsy,html}

\setlength{\topmargin}{-0.2in}

\setlength{\parindent}{0.5cm}

\setlength{\parskip}{0.2cm plus 0.5ex minus 0.2ex}

\setlength{\partopsep}{0.0truept plus 0.0pt minus 0.0pt}

\setlength{\parsep}{0.0truept plus 0.0pt minus 0.0pt}

\setlength{\topsep}{0.0cm plus 0.0pt minus 0.0pt}

\addtolength{\hoffset}{-1.5truecm}

\addtolength{\textwidth}{2.8truecm}

\addtolength{\textheight}{3.9truecm}

\addtolength{\headheight}{-13truept}

\addtolength{\headsep}{-19truept}

\addtolength{\footskip}{20truept}

\date{}



\begin{document}

{\large Proposition de stages de DEA interdisciplinaires : math{\'e}matiques et biologie mol{\'e}culaire }

encadrant : O.Radulescu, IRMAR, Universit\'e de Rennes 1

Depuis quelques ann{\`e}es, une demande croissante de carri{\`e}res 
interdisplinaires se fait sentir dans les
domaines de la science, des
technologies et de la sant{\'e}. Ces propositions font partie d'un programme plus vaste 
visant {\`a} encourager les jeunes scientifiques s'orienter vers de sujets
{\`a} la fronti{\`e}re entre les math{\'e}matiques, la physique et la biologie.


Quelques ph{\'e}nom{\`e}nes
sont incontournables pour la mod{\'e}lisation des organismes biologiques~: 

\begin{enumerate}
\item 
Existence de plusieurs {\'e}chelles de temps.    
\item
Complexit{\'e} des syst{\`e}mes, dont la ma{\^\i}trise impose d'avoir une approche
hi{\'e}rarchique et modulaire.
\item
Stochasticit{\'e} des processus biochimiques conduisant d'une part {\`a} des observations
bruit{\'e}es et d'autre part {\`a} des fluctuations et {\`a} des instabilit{\'e}s.
\end{enumerate}

Le projet MathResoGen (coordonn{\'e} par IRMAR et en collaboration 
avec des {\'e}quipes de biologistes et informaticiens) 
vient de demarrer en novembre 2003. Dans le cadre de ce projet
on se propose de developper 
des nouveaux outils math{\'e}matiques 
pour {\'e}tudier les
ph{\'e}nom{\`e}nes {\'e}num{\'e}r{\'e}s dans des applications 
biologiques concr{\`e}tes.
Deux sujets de stages interdisciplinaires sont propos{\'e}s:

\begin{enumerate}
\item
\htmladdnormallink{Dynamique stochastique de r{\'e}seaux de g{\`e}nes}{../../stage2004.pdf} : mod{\'e}lisation par processus 
de Markov de sauts, homog{\'e}n{\'e}isation, perturbations singuli{\`e}res.
\item
\htmladdnormallink{Interaction g{\`e}nes-m{\'e}tabolisme}{../stage2/stage2.html} : mod{\'e}lisation par {\'e}quations diff{\'e}rentielles ordinaires,
perturbations singuli{\`e}res, stabilit{\'e}.
\end{enumerate}

Les stages  pouront continuer avec des th{\'e}ses. Les candidats devront se faire
conna{\^\i}tre rapidement pour constituer un dossier 
de demande de financement de th{\`e}se.  

Plus de details sur ces sujets  au 

http://name.math.univ-rennes1.fr/ovidiu.radulescu/stages/phd/phd.html

ou par courriel : ovidiu.radulescu@univ-rennes1.fr

\end{document}

