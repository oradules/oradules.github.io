\documentclass[12pt,a4paper]{article}

\usepackage{epsfig,amsmath,amssymb,amsbsy,html}
\usepackage[latin1]{inputenc}

\setlength{\topmargin}{-0.2in}

\setlength{\parindent}{0.5cm}

\setlength{\parskip}{0.2cm plus 0.5ex minus 0.2ex}

\setlength{\partopsep}{0.0truept plus 0.0pt minus 0.0pt}

\setlength{\parsep}{0.0truept plus 0.0pt minus 0.0pt}

\setlength{\topsep}{0.0cm plus 0.0pt minus 0.0pt}

\addtolength{\hoffset}{-1.5truecm}

\addtolength{\textwidth}{2.8truecm}

\addtolength{\textheight}{3.9truecm}

\addtolength{\headheight}{-13truept}

\addtolength{\headsep}{-19truept}

\addtolength{\footskip}{20truept}

\date{}



\begin{document}




  PROPOSITION DE STAGE POUR LE D.E.A

  "Application des Math�matiques et de l'Informatique � la Biologie"

    TITRE DU STAGE :
{\large 
Interaction g{\`e}nes-m{\'e}tabolisme : mod{\'e}lisation par {\'e}quations diff{\'e}rentielles ordinaires,
perturbations singuli{\`e}res, stabilit{\'e}.
 }

    DESCRIPTION DU SUJET DE STAGE PROPOSE :

Les {\'e}tudes biologiques montrent
qu'aussi pour des organismes uni-cellulaires que pour d'organismes pluricellulaires 
les g{\`e}nes ont un r{\^o}le important dans le  fonctionement du m{\'e}tabolisme. Les variables m{\'e}taboliques sont 
constitu{\'e}es par les  concentrations d'enzymes et des produits m{\'e}taboliques (glucose, acides gras, etc.).
L'interaction entre ces variables est decrite par un graphe et la dynamique par un syst{\`e}me
d'{\'e}quations diff\'erentielles non-lin{\'e}aires. 
Les effets du couplage g{\`e}nes/m{\'e}tabolisme sont tr\`es peu \'etudi\'es.  

Le but de ce stage est la mod{\'e}lisation du couplage entre les g{\`e}nes et le m{\'e}tabolisme. Des informations sont disponibles 
pour le m{\'e}tabolisme des lipides dans le foie chez le poulet. On veut construire un mod{\`e}le simple qui explique le
comportement observ{\'e}  en fonction du flux d'entr{\'e}e (absence ou pr{\'e}sence de nourriture) des niveaux de g{\`e}nes et 
des voies m{\'e}taboliques actives. 

Apr{\`e}s l'{\'e}tape de mod{\'e}lisation, l'analyse du mod{\`e}le utilisera des techniques d'�tude des �quilibres,
des m�thodes de perturbations singuli{\`e}res et
la simulation num{\'e}rique.       

Pr{\'e}requis: syst{\`e}mes d'{\'e}quations diff{\'e}rentielles ordinaires, perturbations, programmation en MATLAB ou SCILAB. 

    BIBLIOGRAPHIE DE REFERENCE SUR LE SUJET :

1) E.Duplus, C.Forest, Biochemical Pharmacology, 64 (2002) 893-901.

2) JP.P�gorier, C. Le May, Nutrition Clinique et M�tabolisme, 17 (2003) 80-88.



    INFORMATIONS :

      Adresse du laboratoire o� se d�roule le stage: IRMAR, Universit� de Rennes 1, Campus de Beaulieu, 
      35042 Rennes

      Nom et coordonn�es (email ou tel.) de l'encadrant du stage : 

Ovidiu Radulescu, ovidiu.radulescu@univ-rennes1.fr, 02 23 23 60 23.

      Poursuite en Th�se possible : Oui 

�




\end{document}

