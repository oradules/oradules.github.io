
% expose de Ben a Rennes le 12 mai 2004

\documentclass[oneside,slidesonly]{seminar}
\usepackage[francais]{babel}
\usepackage[latin1]{inputenc}
\selectlanguage{francais}
\usepackage{graphicx}
\usepackage{euscript}
\usepackage{amssymb}
\usepackage{delarray}
\usepackage{amsmath}
\usepackage{array}
\usepackage{epsfig}
\usepackage{changebar}
\selectlanguage{francais}
\usepackage{amsfonts}

%\usepackage{color}

%\definecolor{brown}{rgb}{0.7,0.5,0}
%\definecolor{orange}{rgb}{0.9,0.1,0}

%\newcommand{\green}{\textcolor{green}}
%\newcommand{\red}{\textcolor{red}}
%\newcommand{\blue}{\textcolor{blue}}
%\newcommand{\black}{\textcolor{black}}
%\newcommand{\brown}{\textcolor{brown}}
%\newcommand{\yellow}{\textcolor{yellow}}
%\newcommand{\orange}{\textcolor{orange}}


\newcommand{\N}{\mathbb N}
\newcommand{\R}{\mathbb R}
\newcommand{\Z}{\mathbb Z}
\newcommand{\C}{\mathbb C}
\def\({\left(}
\def\){\right)}
\def\WF{W\negmedspace F}

\def\a{\alpha}
\def\r{\rho}
\def\t{\tau}
\def\b{\beta}
\def\d{\partial}
\def\g{\gamma}
\def\l{\lambda}
\def\s{\sigma}
\def\o{\omega}
\def\p{\varphi}
\def\v{\wedge}
\def\ve{v^{\varepsilon}}
\def\ae{a^{\varepsilon}}
\def\nn{\vec{\nu}}


\def\C{\mathbb{C}}
\def\O{\Omega}
\def\lap{\Delta}
\def\G{\Gamma}
\def\D{{\cal D}}
\def\F{{\cal F}}
\def\n{\vec{n}}
\def\phi{\varphi}
\def\e{\varepsilon}
\def\S{\mathbb{S}}
\def\P{\mathcal{P}}
\def\R{\mathbb{R}}
\def\N{\mathbb{N}}
\def\H{\mathbb{H}}\def\ZZ{\mathbb{Z}}
\def\L{\mathbb{L}}
\def\E{{\cal E}}
\def\V{\mathcal{O}_1}
\def\pt{\forall\;}
\def\T{\Theta}
%Les ensembles:

\def\un{\underline}
\def\div{\mbox{{\rm div}}\;}
\def\rot{\mbox{{\rm curl}}\;}
\def\ddt#1{\frac{\partial #1}{\partial t}}
\def\ddi#1{\frac{\partial #1}{\partial x_i}}
\def\dd#1#2{\frac{\partial #1}{\partial #2}}



\def\op{\d^{-1}_{\mu}}


\def\tiu#1{\widetilde {U^{#1}}}
\def\ovu#1{\overline {U^{#1}}}
\def\tidu#1#2{\widetilde {U_{#1}^{#2}}}
\def\ovdu#1#2{\overline {U_{#1}^{#2}}}
\def\wt{\rightharpoonup}
\def\ds{\longrightarrow}

%le displaystyle

\def\dsp{\displaystyle}


\newtheorem{theo}{Th�or�me}
\newtheorem{lem}{Lemme}
\newtheorem{prop}{Proposition}

\newenvironment{demo}[1][.]{\noindent %
\textbf{Proof\ifthenelse{\equal{#1}{.}}{}{(#1)}. }}%
{\hspace{\stretch{1}}\rule{2mm}{2mm}}

\newenvironment{demo_nv}[1][.]{\noindent %
\textbf{New proof\ifthenelse{\equal{#1}{.}}{}{(#1)}. }}%
{\hspace{\stretch{1}}\rule{2mm}{2mm}}

\slideframe[]{}

\begin{document}


\begin{slide} \begin{center} {\bf Generalized Poincar�-Hopf bifurcation and galloping instabilities in traveling waves} 

 Benjamin Texier, Kevin Zumbrun

 Indiana University \end{center} \end{slide}


\begin{slide} \begin{center} {\bf Plan} \end{center}
 \begin{itemize}
   \item[1.] Combustion, ondes de d\'etonations: ph\'enom\`enes physiques.
   \item[2.] Equations mod\`eles.
   \item[3.] Equations diff\'erentielles avec invariance de groupe: bifurcation de Poincar\'e-Hopf g\'en\'eralis\'ee.
   \item[4.] Application \`a des ondes progressives solutions d'EDP simples. 
  \end{itemize}

\end{slide}


\begin{slide} \begin{center} {\bf Instabilit\'es pour des ondes de d\'etonation}  \end{center} 
  M\'elange explosif gazeux dans un cylindre. Ondes de compression.

 Instabilit\'es:
 \begin{itemize}
 \item[1D:] "galloping waves": variations p\'eriodiques de la vitesse d'une onde de d\'etonation sous l'effet de perturbations longitudinales,
 \item[3D:] "spinning waves":  structure 3D complexe, h\'elicoidale. 
 \end{itemize}

 \vspace{0.3cm}

{\bf But:} Description math\'ematique des "galloping waves", pour des mod\`eles simples.

\end{slide}

\begin{slide} \begin{center} {\bf Equations mod\`eles} \end{center}   \begin{eqnarray}
   \d_t u + \d_x f(u) & = & \d_x( B(u) \d_x u)) + k q \phi(u) z, \nonumber \\
   \d_t z & = & \d_x (D(u, z) \d_x z) - k \phi(u) z. \nonumber 
\end{eqnarray}
 $z$ fraction massique de r\'eactant non br\^{u}l\'e. $\phi$ fonction d'activation. $B, D:$ coefficients de diffusion. $q$ chaleur d\'egag\'ee par la r\'eaction, $k$ taux de r\'eaction. 

Mod\`ele complexe: $u = (\rho, \rho u, e)$ (Navier-Stokes). 

Mod\`ele simplifi\'e: $u$ scalaire (Majda, SIAM J. Appl. Math. 81), $B = 1,\, D = 0.$

Mod\`ele interm\'ediaire: Navier-Stokes avec $B, D = 0:$ pas de diffusion (Zeldovich-von Neumann-Doring 40).

\end{slide}


\end{slide} 

\begin{slide} \begin{center} {\bf Mod\`ele simplifi\'e de Majda} \end{center}
   \begin{eqnarray}
   \d_t u + \d_x f(u) & = & \d_x^2 u + k q \phi(u) z, \nonumber \\
   \d_t z & = & - k \phi(u) z. \nonumber 
\end{eqnarray}
 Hypoth\`eses: $f' > 0,\, f" > 0,$ et
 $$  \phi(u) = \left\{\begin{array}{cc} 0, & u \leq u_i, \\ > 0, & u \geq u^i. \end{array}\right.$$
 Existence d'ondes progressives $u_-, \, u_+,\, s(u_+).$
 D\'etonations: ondes de compression: $u_- > u_+.$
 D\'etonations fortes: choc de Lax:
  $$ f'(u_+) < s < f'(u_-).$$



\end{slide}

\begin{slide} \begin{center} {\bf R\'esultats ant\'erieurs} \end{center}
 \begin{itemize}
 \item Stabilit\'e nonlin\'eaire d'ondes de d\'etonations fortes pour le mod\`ele de Majda pour $q$ petit (Liu-Ying, SIAM J. Math. Analysis 95).
 \item Indice de stabilit\'e $\Gamma$ (bas\'e sur fonction d'Evans, Lyng-Zumbrun 03). Condition n\'ecessaire pour stabilit\'e spectrale:  $$\Gamma > 0.$$
 V\'erifi\'ee par le mod\`ele de Majda.
\end{itemize}


 Instabilit\'es ?

 $\Gamma$ ne d\'etecte que la parit\'e du nombre de valeurs propres de partie r\'eelle positive. 


\end{slide}


\begin{slide} \begin{center} {\bf Position du probl\`eme} \end{center}

 Sur un mod\'ele simple (scalaire, une variable d'espace): d\'ecrire les ondes galopantes comme une bifurcation de type Hopf autour d'une onde progressive de d\'etonation. 

 Hypoth\`ese sur le spectre de l'op\'erateur lin\'earis\'e autour de l'onde progressive: deux valeurs propres complexes conjugu\'ees traversent l'axe imaginaire pur pour une valeur critique d'un petit param\`etre.

 Par ailleurs: invariance par translation: 0 est valeur propre. 

 Dessin spectral \`a compl\'eter. 

\end{slide}


\begin{slide} \begin{center} {\bf Stabilit\'e d'ondes progressives}\end{center} 

 R\'eaction de convection-diffusion-r\'eaction:
\begin{equation} \label{edp}
 \d_t \tilde u = {\cal F}(\e, \tilde u) = \d_x^2 \tilde u + f(\e, \tilde u, \d_x \tilde u).
 \end{equation}
 $\bar u^\e(x)$ onde stationnaire:
 \begin{equation} \label{op}
 {\cal F}(\e, \bar u^\e) = 0.
 \end{equation}

 Invariance de groupe: $\tilde \Phi_\a(v) := v(x - \a).$ 

 Lin\'earisation des \'equations autour de l'onde progressive: $u := \tilde u - \bar u,$ solution de
 \begin{equation} \label{edp2}  \d_t (u,\e) + L(u,\e) = G(\bar u, u, \e).\end{equation} 
 $L$ op\'erateur lin\'earis\'e autour de $\bar u.$ 

 Quelles hypoth\`eses spectrale pour $L$ ? Bifurcation de Poincar\'e Hopf g\'en\'eralis\'ee. Spectre essentiel ?
\end{slide}

\begin{slide} \begin{center} {\bf Remarques sur l'\'equation lin\'earis\'ee} \end{center}

 Invariance de groupe: 
 $$ \Phi_\a u (t,x) := u(t, x + \a) + (\bar u^0(x + \a) - \bar u^0(x)).$$

 0 est valeur propre: 
 $$ \frac{d {\cal F}}{d \tilde u} (\e, \bar u^\e) \d_x \bar u^\e = 0.$$ 

\end{slide}

\begin{slide} \begin{center} {\bf Equation diff\'erentielle avec invariance de groupe}\end{center}
 
 Equation $X' = F(X), X \in \R^n,$ admettant une invariance de groupe (additif) $\Phi_\a:$
 $$ \Phi_\a (\Psi(t, t_0, X_0)) = \Psi(t, t_0, \Phi_\a(X_0)),$$
 $\Psi$ flot de l'\'equation. Hypoth\`ese:
 $$ \frac{d \Phi_\a}{d \a}_{|\a = 0} \neq 0.$$
 
\begin{lem}[TZ] Il existe un diff\'eo local $T$ au voisinage de 0 tel que toute trajectoire $X$ s'\'ecrive $X = T(Y, \a),$ $Y \in \R^{n-1}, \, \a \in \R,$ et 
 $$ Y'(t) = G(Y(t)), \quad \a'(t) = h(Y(t)).$$
\end{lem}

 P.~Olver, Applications of Lie Groups to differential equations, Springer GTM 107.

\end{slide}

\begin{slide} \begin{center} {\bf Exemple} \end{center}

 Equation lin\'eaire $X' = A X,$ trajectoires au voisinage de $(0, \dots, 0,1).$ Changement de variable
 $$ X = (x_1, \dots, x_n) \to (\frac{x_1}{x_n}, \dots, \frac{x_{n-1}}{x_n}, 1).$$
 Le groupe additif $\Phi_\a X = e^\a X$ laisse invariante l'\'equation. 

 Les \'equations pour 
 $$ Y := (\frac{x_1}{x_n}, \dots, \frac{x_{n-1}}{x_n}) \quad \mbox{ et } \a := - \ln x_n$$
 sont d\'ecoupl\'ees. 

 V\'erification par un calcul explicite. 


\end{slide}

\begin{slide} \begin{center} {\bf Bifurcation de Poincar\'e-Hopf dans $\R^2$} \end{center} 
 $X' = F(\e, X),\, X \in \R^2,$ $(\e,0)$ droite de points critiques: $F(\e, 0) = 0.$ Hypoth\`ese: le spectre de $\d_x F(\e,0)$ compos\'e de deux valeurs propres conjugu\'ees $\l(\e), \bar \l(\e) = \g(\e) \pm i \tau(\e),$ telles que
 $$ \gamma(0) = 0, \quad \tau(0) \neq 0, \quad (d\gamma)/(d\e)(0) > 0 .$$
 
\begin{theo}[Poincar\'e-Andronov-Hopf] For $a > 0,$ assez petit, il existe une unique orbite p\'eriodique $X_a$ non triviale de l'\'equation, de taille initiale $a:$ $|X_a(0)| = a,$ pour la valeur $\e(a)$ du param\`etre. La fonction $a \mapsto \e(a)$ est $C^1,$ le signe de $d\e/da$ d\'etermine la stabilit\'e de $X_a.$
\end{theo}  

 Hale, Ko\c{c}ak (Springer). 
\end{slide}

\begin{slide} \begin{center} {\bf Bifurcation de Poincar\'e-Hopf dans $\R^n$} \end{center} 
 $X' = F(\e, X),\, X \in \R^n,$ $(\e,0)$ droite de points critiques: $F(\e, 0) = 0.$ Hypoth\`ese: le spectre de $\d_x F(\e,0)$ contient deux valeurs propres conjugu\'ees $\l(\e), \bar \l(\e) = \g(\e) \pm i \tau(\e),$ telles que
 $$ \gamma(0) = 0, \quad \tau(0) \neq 0, \quad (d\gamma)/(d\e)(0) > 0,$$
 et le reste du spectre est contenu dans le demi-plan $\{ z, \, \Re z < 0\}.$ 
 
\begin{theo}[Poincar\'e-Andronov-Hopf] Il existe un voisinage ${\cal U}$ de 0 dans une sous-vari\'et\'e de $\R^n$ tel que pour tout $X_0 \in {\cal U},$ il existe une unique orbite p\'eriodique avec donn\'ee initiale $X_0.$ 
\end{theo} 

 De plus: stabilit\'e (sur la vari\'et\'e $\Leftrightarrow$ sur $\R^n$) exprim\'ee par une condition effectivement calculable; les seules orbites p\'eriodiques sont celles d\'ecrites ci-dessus. 

\end{slide} 


\begin{slide} \begin{center} {\bf Preuve} \end{center}

 \begin{itemize}
 \item[a)] D\'efinition d'un flot r\'eduit sur une vari\'et\'e centrale dans un voisinage de l'origine. 
 \item[b)] Propri\'et\'es spectrales du flot r\'eduit: elles v\'erifient les hypoth\`eses PAH $\R^2.$
 \item[c)] Examen de la stabilit\'e: sur la vari\'et\'e $\Leftrightarrow$ sur $\R^n.$
\end{itemize}

 Marsden, McCraken (Springer). 

\end{slide}

\begin{slide} \begin{center} {\bf Bifurcation de Poincar\'e-Hopf g\'en\'eralis\'ee dans $\R^n$} \end{center}

 M\^{e}me situation spectrale qu'au th\'eor\`eme pr\'ec\'edent. De plus: 
 \begin{itemize} \item le noyau de $\d_x F(\e,0)$ est non trivial,
 \item invariance de groupe $\Phi_\a,$
 \item $(d\Phi_\a)/(d \a)(\a = 0)$ non nul et non transverse au noyau. 
 \end{itemize}

\begin{theo}[TZ] Il existe un voisinage ${\cal U}$ de 0 dans une sous-vari\'et\'e de $\R^n$ tel que pour tout $X_0 \in {\cal U},$ l'orbite $t \to X(t)$ qui a pour donn\'ee initiale $X_0$ est telle que 
 $$ t \to \Phi_{-\a(t)} X(t) $$
 est p\'eriodique, avec $\a(t) = \a^0 t + \beta(t),$ $\a^0 \in \R$ et $\b$ p\'eriodique.
\end{theo} 

\end{slide} 

\begin{slide}\begin{center} {\bf Preuve} \end{center} 

 \begin{itemize} 
 \item[a)] D\'ecouplage des coordonn\'ees transverses $Y$ et longitudinales $\a$ par le Lemme 1: $Y' = G(\e, Y), \, \a' = h(\e, Y).$ 
 \item[b)]  $\d_x G(\e, 0)$ satisfait les hypoth\`eses de PAH $\R^n.$ Existence d'orbites p\'eriodiques.
 \item[c)] Coordonn\'ee longitudinale correspondante:
 $$ \a(t) = \a(0) + \int_0^t h(\e, Y(t')) \, dt' = \a^0 t + \beta(t).$$
 \item[d)] Dans les variables de d\'epart:
  $$ X(t) = \Phi_{\a(t)} (Y(t), 0). $$
 \end{itemize}


\end{slide}

\begin{slide} \begin{center} {\bf Un r\'esultat classique de stabilit\'e} \end{center}

 \begin{theo} Soit $\bar u$ une onde progressive qui converge exponentiellement vers $u_\pm$ en $\pm \infty.$ Si le spectre de $L(u_\pm)$ est strictement \`a gauche de l'axe imaginaire pur, et si $\bar u$ est monotone, alors $\bar u$ est orbitalement stable. 
 \end{theo}

 Preuve: Spectre essentiel de $L(\bar u)$ $\leftrightarrow$ spectre de $L(u_\pm).$ Monotonie de $\bar u:$ 0 est la plus grande valeur propre. 

\end{slide}

\begin{slide} \begin{center} {\bf Exemple: convection-diffusion} \end{center}

  $$ \d_t u + \d_x (f(u)) = \d_x^2 u,$$
 Onde stationnaire $\bar u.$ RH: $f(u_-) = f(u_+).$ 
 $$ L(\bar u) u  = \d_x^2 u - f'(u_\pm) \d_x u.$$
 $\sigma(L(u_\pm)) = \{ - k^2 - i f'(u_\pm) k, \, k \in \R\}.$ 
 Pas de trou spectral.
\end{slide}

\begin{slide} \begin{center} {\bf Normes \`a poids} \end{center}
 
 Cadre pr\'ec\'edent. Hypoth\`ese: choc de Lax:
 $$ f'(u_+) < 0 < f'(u_-).$$
 Alors il existe un poids $w > 0,$ $w \to w_\pm, \,  x \to \pm \infty,$ $w_- <0, \, w_+ > 0,$ tel que 
  $$ v: = e^{\int_0^x w} u,$$
 et l'op\'erateur pour $v$ a un trou spectral. 

 "Convection-enhanced stability". Sattinger (Adv. Math. 76). 
 \end{slide}


\begin{slide} \begin{center} {\bf Hypoth\`eses} \end{center}

 \begin{itemize}
\item  Les \'etats finaux $u_-, \, u_+$ de l'onde stationnaire sont strictement spectralement stables au sens des normes \`a poids.
\item  Deux valeurs propres complexes traversent l'axe imaginaire pur.
\end{itemize}
 $$ \| f\|_{L^2_\eta} := \| e^{\eta(1 + |x|^2)} f(x) \|_{L^2}.$$ 

\end{slide}

\begin{slide} \begin{center} {R\'esultat} \end{center}

\begin{theo}[TZ] Sous les hypoth\`ese pr\'ec\'edentes, pour $a > 0$ assez petit et $C > 0$ assez grand, il existe une fonction $C^1$ $a \mapsto \e(a),$ et une famille de solutions
 $$ u^a(t, x) = v^a(t, x - \a^a(t)), \quad \a^a(t) = \sigma^a t + \theta^a(t),$$
 o\`u $\sigma^a \in \R$ et $\theta^a$ est p\'eriodique, et 
 $$ \| v(0) - \bar u^0 \|_{H^2_\eta} = a, \quad \| v^a(t)  - \bar u^0 \|_{H^2_\eta} \leq C a.$$
  \end{theo}
 De plus: stabilit\'e orbitale de $v^a$ donn\'e par le signe de $d\e/da.$ 

\end{slide}


\begin{slide} \begin{center} {\bf Preuve} \end{center}

 \begin{itemize}
 \item[a)] Factorisation de la valeur propre 0 (r\'eduction).
 \item[b)] Construction d'une vari\'et\'e centrale (troncature, bornes sur le semi-groupe).
 \item[c)] Bifurcation de Poincar\'e-Hopf sur la vari\'et\'e de dimension 2 + 1. 
 \end{itemize} 

\end{slide}



\end{document}